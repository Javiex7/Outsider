\section{Conclusiones}

La finalidad de este proyecto es, tanto aprender y hacer uso de tecnologías
asíncronas modernas, como la creación de una aplicación totalmente funcional 
que implemente estas tecnologías en detalle.

El objetivo principal es la creación de una aplicación que emule de la forma más
completa el juego de adivinanzas de palabras y roles que se propone. Además de ir implementando
diferentes reglas y funcionalidades al juego base, se proponen otros objetivos secundarios
como la implementación de testing o el despliegue de la aplicación a través de un proveedor de 
servicios cloud. Se quiere profundizar en el uso de websockets y lo que implica su desarrollo, no
simplemente quedarse en la superficie, por ello, todo el desarrollo gira entorno a este eje
central. 

La aplicación final incluye múltiples sistemas interconectados que permiten su correcto
funcionamiento y uso por parte del usuario final a través de un navegador web. Para lograr esto, 
ha sido necesario trabajar con diversas tecnologías/frameworks e implementar la lógica y diseño 
necesarios para crear: Un servidor backend encargado de manejar la lógica websocket y un frontend
responsable de facilitar una interfaz web con todas las funcionalidades pertinentes. 
También ha sido necesario gestionar la comunicación necesaria entre los sistemas así como administrar
los elementos adicionales necesarios.

Para el trabajo relacionado con el backend se destaca el uso complejo y en detalle de tecnologías
websocket a través de Django Channels. Todos los mensajes de juego y la comunicación en tiempo real
entre jugadores no es una implementación sencilla y hay que tener en cuenta muchos 
aspectos de la comunicación asíncrona entre varios usuarios. Además de esta implementación,
se ha realizado trabajo adicional de testing, tanto para aprender el correcto funcionamiento 
de la pruebas automáticas en estos entornos, como para poder probar de forma más 
elaborada la aplicación.

Por otra parte, para el desarrollo de la interfaz web, se han implementado interfaces
mediante el uso de librerías de componentes (Vue y Vuetify) que facilitan mucho el trabajo de desarrollo
a la vez que ofrecen un resultado bastante atractivo. También se destaca toda la lógica de
intercomunicación y el código necesario implementado directamente en el frontend.

Además de la implementación, se destaca la preparación de la aplicación mediante Docker, 
para mejorar su portabilidad, y el despliegue final a través de AWS, 
lo que permite su fácil acceso y uso por parte de cualquier usuario.

Teniendo en cuenta lo explicado en los anteriores párrafos, se puede concluir
que el proyecto cumple los objetivos propuestos a lo largo del desarrollo.



\section{Aspectos pendientes y trabajos futuros}

Teniendo en cuenta la realización de objetivos, existen varios aspectos a tratar en cuanto a la posibilidad
de implementar mejoras o realizar trabajos de ampliación sobre la aplicación.

\subsection{Información de usuarios}
Debido a que no era el objetivo principal del proyecto, no hay ninguna implementación relacionada con la administración
persistente de información sobre los jugadores. Añadir la capacidad a los usuarios para poder registrarse y tener guardada
información sobre sus partidas e información en detalle podría ser interesante, especialmente a la hora de personalizar
un poco más la aplicación.

Sería posible añadir un ranking o tener la capacidad de tener ``amigos'' dentro del juego para poder acceder a partidas entre
usuarios conocidos de forma más sencilla. Los añadidos pueden ser muy diversos, pero todos requerirían un esfuerzo adicional 
para gestionar la comunicación síncrona de una base de datos de algún tipo junto
a la dificultad añadida de implementar un sistema de autenticación de usuarios. 

Por otra parte, todos los elementos adicionales
que se han comentado (como la capacidad de tener una lista de amigos) deberían trabajarse de forma individual.

\subsection{Lista de partidas}
Otro añadido que se muestra interesante, es la capacidad de tener una lista de partidas abiertas para que un jugador se pueda
unir. Este añadido permitiría a los jugadores que quieran jugar con desconocidos unirse de forma directa a una partida 
sin hacer uso de otros medios.

Con esta lista de partidas, sería ideal añadir la capacidad de crear una sala privada o una sala pública. Mientras que a una sala
pública podría acceder cualquier usuario desde esta lista, una sala privada no aparecería en este listado (en el caso de existir muchas
salas de juego) o en todo caso solo sería accesible con un código de acceso.

\subsection{Opciones de partida}
Relacionado con el anterior punto, la capacidad de modificar el estado de la sala y la partida por parte de los usuarios se planteó 
como un posible añadido para la aplicación. Por ejemplo, definir el número de jugadores Outsider o limitar el número de rondas
por partida.

Esta implementación, debido a la carga de lógica que acarrea no se llegó a implementar, pero se destaca que podría ser un gran
añadido y no requeriría la implementación de sistemas nuevos. Sería necesario guardar esta configuración de partida
e implementar una lógica más compleja que tenga en cuenta todas estas variaciones de juego posibles. Ayudaría bastante seguir haciendo
uso del sistema de testing para comprobar de forma más rápida todos estos nuevos cambios.

\subsection{Escalabilidad}
Además del testing implementado, es importante poner a prueba las capacidades de rendimiento de los sistemas. Podría ser
muy interesante plantear un uso masivo de la aplicación y como se tendría que modificar el entorno para adaptarse a estos cambios. Ya que, 
puede que se llegué a un punto donde los servidores se vean sobrepasados o en todo caso, si se sigue haciendo uso de una máquina
EC2 y de AWS, se genere un sobrecoste importante del uso de los servicios de cloud.

En el proyecto no se han realizado pruebas de este tipo y el simple hecho de plantear este tipo de problemas es bastante atractivo
si se quieren explorar estos conceptos de rendimiento y capacidad de carga en un sistema informático.
