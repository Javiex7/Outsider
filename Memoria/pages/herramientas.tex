A continuación se expondrá brevemente el uso de tecnologías así como la metodología
de trabajo que se ha seguido para la elaboración de la aplicación.

\section{Herramientas y tecnologías usadas}

\subsection{Django - Django channels}

Las bases del proyecto se fundamentan en el uso de Django \cite{django} como tecnología de backend.

Además de conocer en profundidad cómo funcionan este framework y Python (que es el lenguaje con el que trabaja Django),
esta se trata de una herramienta que permite un desarrollo rápido, seguro y escalable.

Por otra parte se trabajará con Django channels \cite{djangoChannels} para gestionar el uso de websockets. Django channels
permite el uso de websockets y tecnologías análogas mediante varios paquetes\tutor{librerías?} que se integran dentro del framework de Django.
A través del uso de 'consumers' o consumidores, abstracciones propias de Channels, 
se desarrollará la mayor parte de la lógica de la aplicación. 

Posteriormente se hará mención a pytest \cite{pytest} a la hora de implementar el testing automático.

\subsection{Vue3}

Por otro lado, para el desarrollo frontend se propone el uso de Vue \cite{vue3} debido a su popularidad, versatilidad
y a su uso personal en otros proyectos. De esta forma se creará un frontend sencillo pero bastante personalizable.

Se hace referencia a Vue3 por ser la versión más moderna de Vue, la cual tiene cambios destacables en comparación
a Vue2 \cite{vue3vue2}. Todos los paquetes y librerías adicionales se incorporarán teniendo en cuenta que se está haciendo uso de
Vue3. En especial, se destaca la librería de componentes Vuetify \cite{vuetify}, la cual es un gran añadido a la hora de
ofrecer una gran diversidad de componentes para la construcción de una interfaz vistosa y dinámica.

\subsection{Herramientas de diseño}

Figma \cite{figma} es un editor de gráficos que se usa principalmente para generar prototipos. Se usará principalmente para
el diseño y experimentación inicial de la aplicación web. Además de Figma, se harán uso de otras herramientas
menores de diseño como draw.io \cite{draw.io} para la creación de diagramas, Pictogrammers \cite{pictogrammers} para hacer uso de
iconos de forma sencilla o Contrast Finder \cite{contrastFinder} para comparar contrastes de color.

\subsection{Docker}

Docker \cite{docker} es una herramienta de código abierto que facilita un despliegue seguro\tutor{es menos seguro que una VM. Para no pillarte los dedos, usa otros adjetivos como sencillo o popular} y portátil\tutor{portable? portátil no es el adjetivo aquí} utilizando contenedores
virtuales. Estos contenedores se emplearán para hospedar los diversos servicios del sistema en un
servidor.

\subsection{AWS}

Amazon Web Services (AWS)~\cite{aws} es una plataforma de servicios en la nube ofrecida por Amazon. Proporciona
una amplísima gama de servicios que incluyen computación, almacenamiento, bases de datos, ... Estas soluciones permitirán
alojar la aplicación final de forma flexible y sencilla.

Se detallará el uso tanto de Docker como de AWS en el proceso de preparación y despliegue de la aplicación.

\subsection{Github}

GitHub \cite{github} es una plataforma de desarrollo colaborativo basada en la web que utiliza Git, un sistema de control de versiones
distribuido. Permite a los desarrolladores alojar, gestionar y compartir proyectos de software, facilitando la accesibilidad
del código.

Todo el código del proyecto será accesible mediante un repositorio público en Github.

\subsection{WSL}

Se destaca de forma adicional el uso del subsistema de Windows para Linux (WSL) \cite{wsl} para poder ejecutar de forma sencilla
todo el entorno de la aplicación así como para la realización de pruebas de despliegue sin dejar de hacer uso directo de Windows
como sistema operativo.

\subsection{Visual Code}

En último lugar se encuentra la herramienta más básica y a la vez más importante para
el proyecto, Visual Studio Code \cite{vscode}. Visual Studio Code, comúnmente conocido como Visual Code o VS Code,
es un editor de código fuente desarrollado por Microsoft.

Todo el código se ha tratado mediante Visual code, desde la creación y edición del código hasta la ejecución de comandos
de terminal.


\section{Metodología}

Este proyecto se ha guiado por una metodología de trabajo iterativa-incremental\tutor{te añado yo esto, para ponerle nombre a lo que hemos hecho}, que se enfoca en la revisión de objetivos y
avances entre el tutor y el alumno. Durante cada encuentro, se han ido discutiendo propuestas de mejora que se han convertido en
objetivos para la siguiente evaluación. En primera instancia se han planteado unos objetivos principales (crear el juego
con funcionalidades básicas) y diversos objetivos adicionales/opcionales que se han ido discutiendo e
implementando a lo largo de las diferente reuniones (testing, mecánicas adicionales, despliegue mediante AWS, ...).

Esta metodología ha demostrado ser efectiva para este tipo de proyecto, ya que permite un avance progresivo y detallado,
evitando la acumulación de muchas tareas. Además, al planificar con visión a futuro, la aplicación ha crecido de manera sólida
y constante, sin necesidad de realizar cambios radicales.
