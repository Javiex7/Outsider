Este documento aborda y explica el desarrollo de este proyecto de fin de grado,
desde su concepción inicial y las motivaciones que lo impulsaron, hasta los detalles
finales de su implementación.

\section{Descripción general}

El objetivo general del proyecto es el desarrollo de un juego multijugador web de adivinanza de palabras y roles ocultos. El principal punto de interés
radica en el uso de tecnologías en tiempo real, específicamente de tecnologías websocket, posibilitando el juego online entre los jugadores
de forma fiable y óptima.

Las mecánicas base del juego son sencillas y son más una adaptación de un tipo de juego que normalmente se realiza sin el uso de tecnologías informáticas.
El concepto más básico del juego se podría resumir mediante los siguientes puntos:

\begin{enumerate}
	\item En este juego, todos los jugadores juegan como ``Inocentes'', a excepción de uno de ellos,
	      definido como el ``Outsider''.
	\item Al empezar la partida, los jugadores
	      Inocentes reciben una
	      palabra secreta o contraseña, como por ejemplo, ``Hoja''.
	\item  Siguiendo un orden secuencial (con el primer jugador elegido al azar), cada
	      participante debe escribir una palabra relacionada con su palabra clave
	      para indicar a los demás que conocen la palabra asignada.
	\item Por ejemplo si la palabra clave es la palabra ``Hoja'', unas palabras que
	      alienten a esta  o palabra clave podrían ser ``Árbol'', ``Libro'' o ``Afeitado''.
	\item El jugador designado como Outsider, quien no tiene conocimiento de la palabra secreta, debe tratar de deducirla a partir de las palabras
	      previamente mencionadas y decir una palabra que no levante sospechas.
	\item Después de que todos los jugadores hayan compartido una palabra, se lleva a cabo una votación simultánea, donde cada jugador vota por la persona que creen
	      que es el Outsider.
	\item El Outsider ganará si no es el jugador más votado. Por otra parte, el resto de los jugadores Inocentes ganarán
	      si logran descubrir al Outsider y este llega a ser el jugador más votado.
\end{enumerate}

Dadas estas reglas de funcionamiento del juego, el desarrollo del proyecto consiste principalmente en
interpretar y diseñar una aplicación web capaz de poder gestionar todos estos puntos. En el próximo capítulo de Objetivos (\ref{Objetivos}),
se tratará con más detalle esta definición de reglas.

Además de generar una aplicación/juego que emule estas reglas, es bastante destacable el trabajo adicional relacionado con la gestión informática que
implica desarrollar una aplicación web: desde el diseño de interfaces usables hasta su despliegue y testeo, todos estos apartados se trataran en detalle
como parte del desarrollo.

\section{Motivación}

La principal motivación para la realización de este proyecto es el aprendizaje propio. La idea del juego
surge por parte del tutor del TFG, \nombretutor, que ve muy interesante la creación de una aplicación que haga uso de tecnologías en
tiempo real. También ha sido una figura clave en el desarrollo iterativo del proyecto, proponiendo mejoras y comentarios
a las diferentes versiones de la aplicación.

Dicho esto, a día de hoy hay innumerables proyectos, aplicaciones y juegos que persiguen conseguir un rendimiento económico o lúdico, sin embargo, también es importante destacar el papel
de la investigación y el aprendizaje, especialmente en el ámbito académico en el que se sitúa este trabajo de fin grado.

Las empresas y organizaciones buscan desarrolladores software altamente capacitados, pero también especializados en materias en concreto.
Destaca más un desarrollador que sepa sobre una tecnología útil y difícil de aprender sobre otro que sepa
sobre otra tecnología menos útil o mucho más accesible. Aquí es donde entran en juego los websockets, cuyas implementaciones suelen dar problemas 
a los desarrolladores.

Muchos estudiantes hemos realizado pequeños proyectos y pruebas con websockets y tecnologías análogas, pero es muy diferente desarrollar toda una aplicación, realizar testing e
incluso llevar a cabo un despliegue en torno a tecnologías en tiempo real. Hay que tener en cuenta muchas variables y por otra parte se tratan de tecnologías útiles
con infinidad de aplicaciones. Desde un simple chat en línea al uso de websockets en grandes proyectos, el uso de aplicaciones web en tiempo real es muy práctico.

Desde mi experiencia previa en desarrollo de aplicaciones móviles, un desarrollador no es consciente de la cantidad de lógica que se necesita para manejar una base
de datos en lo que a priori se presenta como una sencilla aplicación. Todo el trabajo subyacente es muy poco conocido pero suele ser de gran importancia. En este trabajo previo,
se destaca que se planteó varias veces la necesidad de poder crear conexiones websockets entre usuarios de la aplicación para poder compartir datos de forma inmediata.

Las aplicaciones que usamos en nuestros teléfonos y dispositivos, los videojuegos online o los servicios de streaming 
dependen de una estructura muy bien pensada para cada caso específico. Por ello, es una necesidad por parte de un desarrollador software, 
al menos entender, como funcionan todos estos sistemas. Mi objetivo es explorar y aprender lo máximo de la mayor cantidad de tecnologías 
software para desarrollarme como profesional así como para enseñar las posibilidades que ofrecen estas diversas herramientas.
