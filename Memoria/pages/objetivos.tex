En esta sección se definirán los objetivos principales del proyecto así como de otros objetivos y
requisitos secundarios que han ido definiéndose a lo largo del proyecto.

\section{Objetivos funcionales del juego}

En primera instancia destacamos las funcionalidades básicas que debería cumplir el juego como
producto software, es decir, relacionados con el usuario de forma directa.

\subsection{Funcionalidad básica, lobby y conexiones}

Además del propio juego, es necesario dar las capacidades y herramientas necesarias a los usuarios para
poder organizar partidas de forma sencilla. Para ello se plantea la creación de una página de inicio que
permita al usuario poder crear salas de juego o unirse a salas ya creadas.

El objetivo principal sería permitir a todos los usuarios de la aplicación poder jugar a
diferentes partidas teniendo en cuenta varias limitaciones como el que un jugador no debería poder acceder
a una partida ya empezada o evitar la creación de salas con el mismo nombre. De forma adicional, en esta página se deberían mostrar instrucciones que expliquen el funcionamiento
del juego.

También sería esencial tener una pantalla pre-partida en la cual se vayan uniendo los jugadores
antes de comenzar el juego per se. En esta pantalla se propone añadir un chat para los jugadores además
de mostrar la información pertinente al estado de la partida: Número de jugadores preparados, código de la
sala para que se puedan unir más jugadores, ...

\subsection{Juego sencillo}

En primer lugar, se plantea poder jugar de forma sencilla solo una ronda en la cual se deciden los
roles de los jugadores, se les indica una pista/contraseña a los jugadores Inocentes y se procede con la lógica
de juego estándar: Cada jugador siguiendo un orden secuencial, escribirá una palabra relacionada con su contraseña
para indicar a los demás que conocen la palabra asignada; después de que todos los jugadores hayan compartido una
palabra, se lleva a cabo una votación simultánea, donde cada jugador vota por la persona que creen
que es el Outsider.

El Outsider ganará si no es el jugador más votado. Por otra parte, el resto de los jugadores Inocentes ganarán
si logran descubrir al Outsider y este llega a ser el jugador más votado.

El objetivo es llevar a cabo la implementación de esta lógica de la forma más consistente posible, trabajando en
detalle con las conexiones websocket.

\subsection{Útima oportunidad}

Habiendo implementado el juego sencillo, se querría añadir nuevas funcionalidades base. En primer lugar, dado el transcurso
normal del juego, tras la votación simultánea, en el caso de que el Outsider reciba la mayoría de los votos, se le otorgará
una última oportunidad para ganar adivinando la contraseña.

\subsection{Multirondas}

Cuando existen más de tres jugadores en una partida, se ve bastante necesario el poder jugar varias rondas con otras contraseñas,
ya que, si no se elimina al Outsider de primeras y se elimina a un jugador Inocente, podrían seguir jugando el resto de jugadores
rondas adicionales.

Los jugadores eliminados pasan a ser espectadores mientras el resto sigue jugando hasta que el número de jugadores sea
igual al de Outsiders o se elimine mediante votación al Outsider en cuestión.

\subsection{Varios Outsider}

Relacionado con el anterior punto, en partidas con varios jugadores se ve necesario aumentar el número de Outsiders en aras de
una jugabilidad más interesante. De esta manera, se plantea que cuando el número de jugadores sea mayor que 6, simplemente sean
dos jugadores Outsider en la partida.

En estos casos el Outsider solo tendrá la oportunidad de adivinar la contraseña si es el último Outsider en la partida.

\section{Objetivos secundarios y técnológicos}

Dados los objetivos/reglas principales que debería cumplir el juego, se añaden varios puntos adicionales que se quieren
tratar como requisitos del proyecto.

\subsection{Plataforma de juego}

Se quiere aprovechar el uso de un entorno web para hacer más accesible el juego a diferente tipo de usuarios.

El objetivo es hacer usable la aplicación en el mayor número de dispositivos posibles y se quiere diferenciar principalmente entre
teléfonos móviles y equipos de escritorio (ordenadores y portátiles principalmente). Se diferencia entre estos dos tipos de
dispositivos porque también se preveen dos tipos de juego de forma predominante:

\begin{enumerate}
    \item Juego presencial en un mismo espacio físico
    \item Juego remoto
\end{enumerate}

Se entiende que si un grupo quiere jugar en un espacio físico común, por ejemplo, en un cumpleaños en la casa de uno de los
jugadores, en la mayoría de casos todos los jugadores jugarán con sus propios dispositivos móviles, principalmente smartphones.

Por otro lado, también se ve bastante común que otro grupo de jugadores quiera jugar de forma remota, cada uno en una
localizacón diferente. En este caso, se podría asumir que la mayoría de jugadores intentarían jugar con sus dispositivos de
escritorio.

\subsection{Testing}

Se propone de forma adicional trabajar ciertos elementos de CI, especialmente el testing del software, ya que, es una práctica
esencial en la industria y resulta interesante trabajar el testing con tecnologías en tiempo real.

\subsection{Despliegue}

Para poder enseñar el resultado del proyecto de la forma más accesible posible, también se querría trabajar con tecnologías de
AWS (Amazon Web Services) para poder tener la aplicación desplegada y totalmente disponible para cualquier usuario a la hora
de acceder a esta.

De esta forma se pretende aprender sobre tecnologías de despliegue y de crear una mejor presentación final del proyecto.

\subsection{Feedback de usuarios}

Para finalizar con la descripción inicial de objetivos/requisitos, se querría tener en cuenta una pequeña prueba con usuarios a mitad de
desarrollo con el objetivo de encontrar bugs y pequeñas mejoras para el juego, lo que dará su propia lista de cambios a realizar.

\section{Resumen de objetivos}

\begin{enumerate}
    \item Gestionar las conexiones entre jugadores mediante salas.
    \item Creación del juego básico.
    \item Añadir una mecánica adicional a la hora de eliminar al último jugador Outsider.
    \item Permitir jugar varias rondas.
    \item Añadir más jugadores Outisder al juego si hay muchos jugadores en partida.
    \item Gestionar la usabilidad de la aplicación en diferentes dispositivos.
    \item Tener en cuenta el feedback de los usuarios - Testing manual.
    \item Realiación de tests para la lógica en tiempo real - Testing automático.
    \item Desplegar la aplicación a través de tecnologías modernas.
\end{enumerate}